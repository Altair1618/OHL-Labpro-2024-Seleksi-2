\documentclass{article}

\usepackage{geometry}
\usepackage{amsmath}
\usepackage{graphicx, eso-pic}
\usepackage{listings}
\usepackage{hyperref}
\usepackage{multicol}
\usepackage{fancyhdr}
\pagestyle{fancy}
\fancyhf{}
\hypersetup{ colorlinks=true, linkcolor=black, filecolor=magenta, urlcolor=cyan}
\geometry{ a4paper, total={170mm,257mm}, top=10mm, right=20mm, bottom=20mm, left=20mm}
\setlength{\parindent}{0pt}
\setlength{\parskip}{0.3em}
\renewcommand{\headrulewidth}{0pt}

\rfoot{\thepage}
\fancyhf{} % sets both header and footer to nothing
\renewcommand{\headrulewidth}{0pt}
\lfoot{\textbf{OHL Labpro 2024}}
\pagenumbering{gobble}

\fancyfoot[CE,CO]{\thepage}
\lstset{
    basicstyle=\ttfamily\small,
    columns=fixed,
    extendedchars=true,
    breaklines=true,
    tabsize=2,
    prebreak=\raisebox{0ex}[0ex][0ex]{\ensuremath{\hookleftarrow}},
    frame=none,
    showtabs=false,
    showspaces=false,
    showstringspaces=false,
    prebreak={},
    keywordstyle=\color[rgb]{0.627,0.126,0.941},
    commentstyle=\color[rgb]{0.133,0.545,0.133},
    stringstyle=\color[rgb]{01,0,0},
    captionpos=t,
    escapeinside={(\%}{\%)}
}

\begin{document}

\begin{center}

    
    \section*{Chipipi Chapapa}

    \begin{tabular}{ | c c | }
        \hline
        Batas Waktu  & 1s \\
        Batas Memori & 64MB \\
        \hline
    \end{tabular}

    % Tag: OOP Konseptual
    % Difficulty: Easy
    
\end{center}

\subsection*{Deskripsi}

Purry sedang mempelajari OOP menggunakan bahasa Chipipi (C++). Ia pun menemukan sebuah teka-teki yang berjudul "Chipipi Chapapa". Dalam teka-teki ini, terdapat 6 kelas sesuai yang ada pada kode \href{https://drive.google.com/file/d/14HL7qtTQEweuAHRCAryBxWxXb4_PrH0d/view?usp=drive_link}{berikut}. Simpelnya, diberikan dua bilangan bulat positif, yaitu \textit{m} dan \textit{n}. Purry diminta untuk membuat program yang dapat menghasilkan luaran berupa \(-m\) dan \(-2n\) serta sebuah \textit{string custom} "Wrapper". Namun, Purry hanya dapat menggunakan apa yang sudah ada pada kode tersebut.

\subsection*{Format Masukan}

Masukan berupa 1 baris yang berisi dua integer positif \textit{m} dan \textit{n} dengan format \textit{m} \textit{n}, dengan batas \(1 \leq m, n \leq 100\).

\subsection*{Format Keluaran}

Keluaran berupa 2 baris dengan baris pertama adalah \(-m\) \(-2n\) dan baris kedua adalah \textit{string} "Wrapper".

\begin{multicols}{2}
\subsection*{Contoh Masukan}
\begin{lstlisting}
1 2
\end{lstlisting}
\columnbreak
\subsection*{Contoh Keluaran}
\begin{lstlisting}
-1 -4
Wrapper
\end{lstlisting}
\vfill
\null
\end{multicols}


\subsection*{Penjelasan}

\(a = 1\) dan \(b = 2\), maka \(-a = -1\) dan \(-2b = -4\). Luaran \textit{string} "Wrapper" dikeluarkan secara \textit{custom} lewat program.

\subsection*{Catatan}

\begin{enumerate}
    \item Salin seluruh kode, lalu tulis perubahan hanya pada bagian yang telah disediakan di fungsi \(main()\). \textbf{Dilarang memodifikasi kode yang ada di atasnya!}
    \item Untuk melakukan \textit{output}, gunakan fungsi/metode yang telah dituliskan pada kode (seperti \(B::print()\)). Anda dilarang untuk menggunakan fungsi/metode lain untuk melakukan \textit{output} (seperti \(printf()\) dan \(cout <<\)) di dalam fungsi \(main()\). 
    \item Dilarang menggunakan \textit{scope resolution operator} (::). Silakan manfaatkan konsep \textit{inheritance} dan \textit{polymorphism}.
\end{enumerate}

\end{document}
