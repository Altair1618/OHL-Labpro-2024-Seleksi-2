\documentclass{article}

\usepackage{geometry}
\usepackage{amsmath}
\usepackage{graphicx, eso-pic}
\usepackage{listings}
\usepackage{hyperref}
\usepackage{multicol}
\usepackage{fancyhdr}
\pagestyle{fancy}
\fancyhf{}
\hypersetup{ colorlinks=true, linkcolor=black, filecolor=magenta, urlcolor=cyan}
\geometry{ a4paper, total={170mm,257mm}, top=10mm, right=20mm, bottom=20mm, left=20mm}
\setlength{\parindent}{0pt}
\setlength{\parskip}{0.3em}
\renewcommand{\headrulewidth}{0pt}

\rfoot{\thepage}
\fancyhf{} % sets both header and footer to nothing
\renewcommand{\headrulewidth}{0pt}
\lfoot{\textbf{OHL Labpro 2024}}
\pagenumbering{gobble}

\fancyfoot[CE,CO]{\thepage}
\lstset{
    basicstyle=\ttfamily\small,
    columns=fixed,
    extendedchars=true,
    breaklines=true,
    tabsize=2,
    prebreak=\raisebox{0ex}[0ex][0ex]{\ensuremath{\hookleftarrow}},
    frame=none,
    showtabs=false,
    showspaces=false,
    showstringspaces=false,
    prebreak={},
    keywordstyle=\color[rgb]{0.627,0.126,0.941},
    commentstyle=\color[rgb]{0.133,0.545,0.133},
    stringstyle=\color[rgb]{01,0,0},
    captionpos=t,
    escapeinside={(\%}{\%)}
}

\begin{document}

\begin{center}

    
    \section*{Pohon B Plus} % ganti judul soal

    \begin{tabular}{ | c c | }
        \hline
        Batas Waktu  & 1s \\    % jangan lupa ganti time limit
        Batas Memori & 256MB \\  % jangan lupa ganti memory limit
        \hline
    \end{tabular}
\end{center}

\subsection*{Deskripsi}

Dr. Agus Heisenberg sedang merancang sebuah basis data untuk menyimpan data-data agen OWCA. Seiring meningkatnya jumlah agen OWCA, proses pencarian agen pada basis data tersebut semakin lambat. Lalu, Dr. Agus Heisenberg menemukan bahwa proses pencarian dapat dipercepat dengan struktur data B+ Tree. 

Sebagai agen dengan kemampuan programming yang handal, Dr. Agus memintamu untuk membuatkan ADT B+ Tree untuknya. Karena data agen lama tetap disimpan, B+ Tree yang dibuat cukup mengimplementasi fitur insertion. Dr. Agus telah meringankan tugasmu dengan membuatkan template program pada link \href{https://drive.google.com/drive/folders/19CcSmHoxypupEgl6w9DckooEcUfqM3qY?usp=sharing}{berikut}

Bantulah Dr. Agus dengan melengkapi program ADT B+ Tree yang telah dia berikan.

Visualisasi proses insertion B+ Tree dapat mereferensi pada website \href{https://www.cs.usfca.edu/~galles/visualization/BTree.html}{berikut}

\subsection*{Format Masukan}

Baris pertama terdiri dari dua bilangan bulat positif $D$ ($1 \leq D \leq 10$) dan $N$ ($1 \leq N \leq 10^{5}$) yang masing-masing menyatakan derajat pohon dan jumlah operasi.

$N$ baris berikutnya berisi dua bilangan bulat positif $O$ ($1 \leq O \leq 2$) dan $K$ ($1 \leq N \leq 10^{9}$) yang masing-masing menyatakan operasi yang dilakukan dan nilai operasi tersebut

\subsection*{Format Keluaran}

Format keluaran sudah ditentukan dalam program template pada link \href{https://drive.google.com/drive/folders/19CcSmHoxypupEgl6w9DckooEcUfqM3qY?usp=sharing}{berikut}

\begin{multicols}{2}
\subsection*{Contoh Masukan 1}
\begin{lstlisting}
5 7
1 7
1 9
2 13
1 13
1 15
1 8
2 7
\end{lstlisting}
\columnbreak
\subsection*{Contoh Keluaran 1}
\begin{lstlisting}
Not Found
1
Found
B+ Tree Akhir:
-7
-8
9
-9
-13
-15
\end{lstlisting}
\vfill
\null
\end{multicols}


\subsection*{Penjelasan}

B+ Tree yang terbentuk seperti pada gambar \href{https://drive.google.com/file/d/1pot9zNXOrSjS-J8ebY-b_D8aXAC7xbT4/view?usp=sharing}{berikut}

\end{document}