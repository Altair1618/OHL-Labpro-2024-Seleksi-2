\documentclass{article}

\usepackage{geometry}
\usepackage{amsmath}
\usepackage{graphicx, eso-pic}
\usepackage{listings}
\usepackage{hyperref}
\usepackage{multicol}
\usepackage{fancyhdr}
\pagestyle{fancy}
\fancyhf{}
\hypersetup{ colorlinks=true, linkcolor=black, filecolor=magenta, urlcolor=cyan}
\geometry{ a4paper, total={170mm,257mm}, top=10mm, right=20mm, bottom=20mm, left=20mm}
\setlength{\parindent}{0pt}
\setlength{\parskip}{0.3em}
\renewcommand{\headrulewidth}{0pt}

\rfoot{\thepage}
\fancyhf{} % sets both header and footer to nothing
\renewcommand{\headrulewidth}{0pt}
\lfoot{\textbf{OHL Labpro 2024}}
\pagenumbering{gobble}

\fancyfoot[CE,CO]{\thepage}
\lstset{
    basicstyle=\ttfamily\small,
    columns=fixed,
    extendedchars=true,
    breaklines=true,
    tabsize=2,
    prebreak=\raisebox{0ex}[0ex][0ex]{\ensuremath{\hookleftarrow}},
    frame=none,
    showtabs=false,
    showspaces=false,
    showstringspaces=false,
    prebreak={},
    keywordstyle=\color[rgb]{0.627,0.126,0.941},
    commentstyle=\color[rgb]{0.133,0.545,0.133},
    stringstyle=\color[rgb]{01,0,0},
    captionpos=t,
    escapeinside={(\%}{\%)}
}

\begin{document}

\begin{center}

    
    \section*{Sipaling Coquette}

    \begin{tabular}{ | c c | }
        \hline
        Batas Waktu  & 1s \\
        Batas Memori & 64MB \\
        \hline
    \end{tabular}
\end{center}

\subsection*{Deskripsi}

Purry akhir-akhir ini sangat menyukai gaya \textit{coquette}. Semuanya ia jadikan berbentuk pita. Hingga suatu hari, Purry mendapatkan tugas untuk menganalisis \textit{n} data titik koordinat (\textit{x}, \textit{y}) yang menyatakan lokasi musuh-musuh Purry. Alih-alih bekerja, Purry lebih memilih untuk menyusun Pita Coquette dari titik-titik ini.\newline
Pita Coquette adalah pita yang memiliki ketentuan sebagai berikut.
\begin{enumerate}
    \item Terbentuk dari 4 titik
    \item Luas area \(> 0\)
    \item Berbentuk simetris, dengan rasio panjang dan tinggi sebesar 2 : 1
    \item Bagian pinggirnya sejajar dengan sumbu y
\end{enumerate}

Untuk lebih jelasnya, dapat lihat ilustrasi \href{https://drive.google.com/file/d/1eYyaiV1-mpoYWLuBTnvTJsZTsRzAdQep/view?usp=drive_link}{berikut}.

Karena tidak bisa menghitung, Purry meminta bantuan Anda. Purry akan memberikan sebuah koordinat target kepada Anda. Anda diminta untuk menghitung banyak Pita Coquette yang dapat dibentuk dengan titik-titik yang ada pada data koordinat musuh, dengan salah satu titiknya berkoordinat sama dengan target.

\subsection*{Format Masukan}

\begin{itemize}
    \item Baris pertama merupakan bilangan bulat \textit{n} yang menyatakan banyak titik (\(4 \leq n \leq 10^5\)).
    \item \textit{n} baris selanjutnya merupakan bilangan-bilangan bulat yang menyatakan data koordinat musuh yang dituliskan dalam bentuk \textit{x y} (\(0 \leq x, y \leq 1000\)). Titik-titik pada data koordinat musuh tidak harus unik. Jika terdapat duplikat, duplikat tersebut dianggap sebagai titik yang berbeda.
    \item Baris selanjutnya adalah koordinat target dalam bentuk \(x_{target}\) \(y_{target}\) (\(0 \leq x_{target}, y_{target} \leq 1000\)). Koordinat target belum tentu ada dalam data koordinat musuh.
\end{itemize}

\subsection*{Format Keluaran}

Sebuah bilangan yang menyatakan banyak Pita Coquette yang dapat dibentuk dengan titik-titik yang ada pada data koordinat musuh, dengan salah satu titiknya berkoordinat sama dengan target. Jika koordinat target tidak ada dalam data koordinat musuh, kembalikan 0.

\begin{multicols}{2}
\subsection*{Contoh Masukan}
\begin{lstlisting}
8
0 0
0 1
2 1
2 0
2 1
0 2
4 2
4 0
0 0
\end{lstlisting}
\columnbreak
\subsection*{Contoh Keluaran}
\begin{lstlisting}
3
\end{lstlisting}
\vfill
\null
\end{multicols}


\subsection*{Penjelasan}

Misalkan kedelapan titik tersebut disimbolkan sebagai 
\(p_0=(0,0), p_1=(0,1), p_2=(2,1), p_3=(2,0), p_4=(2,1), p_5=(0,2), p_6=(4,2), p_7=(4,0)\)
Maka, Pita Coquette yang melalui \((0,0)\) dapat dibentuk melalui titik:
\begin{enumerate}
    \item \(p_0\), \(p_1\), \(p_3\), dan \(p_2\)
    \item \(p_0\), \(p_1\), \(p_3\), dan \(p_4\)
    \item \(p_0\), \(p_5\), \(p_7\), dan \(p_6\)
\end{enumerate}

\end{document}