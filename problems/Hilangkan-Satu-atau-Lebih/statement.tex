\documentclass{article}

\usepackage{geometry}
\usepackage{amsmath}
\usepackage{graphicx, eso-pic}
\usepackage{listings}
\usepackage{hyperref}
\usepackage{multicol}
\usepackage{fancyhdr}
\pagestyle{fancy}
\fancyhf{}
\hypersetup{ colorlinks=true, linkcolor=black, filecolor=magenta, urlcolor=cyan}
\geometry{ a4paper, total={170mm,257mm}, top=10mm, right=20mm, bottom=20mm, left=20mm}
\setlength{\parindent}{0pt}
\setlength{\parskip}{0.3em}
\renewcommand{\headrulewidth}{0pt}

\rfoot{\thepage}
\fancyhf{} % sets both header and footer to nothing
\renewcommand{\headrulewidth}{0pt}
\lfoot{\textbf{OHL Labpro 2024}}
\pagenumbering{gobble}

\fancyfoot[CE,CO]{\thepage}
\lstset{
    basicstyle=\ttfamily\small,
    columns=fixed,
    extendedchars=true,
    breaklines=true,
    tabsize=2,
    prebreak=\raisebox{0ex}[0ex][0ex]{\ensuremath{\hookleftarrow}},
    frame=none,
    showtabs=false,
    showspaces=false,
    showstringspaces=false,
    prebreak={},
    keywordstyle=\color[rgb]{0.627,0.126,0.941},
    commentstyle=\color[rgb]{0.133,0.545,0.133},
    stringstyle=\color[rgb]{01,0,0},
    captionpos=t,
    escapeinside={(\%}{\%)}
}

\begin{document}

\begin{center}

    
    \section*{Hilangkan Satu atau Lebih} % ganti judul soal

    \begin{tabular}{ | c c | }
        \hline
        Batas Waktu  & 2s \\    % jangan lupa ganti time limit
        Batas Memori & 256MB \\  % jangan lupa ganti memory limit
        \hline
    \end{tabular}
\end{center}

\subsection*{Deskripsi}

Anda diberikan sebuah array $A$ yang terdiri dari $N$ bilangan bulat. Anda bisa menghapus paling banyak satu elemen dari array ini. Sehingga, panjang akhir array adalah $N - 1$ atau $N$.

Tugas Anda adalah menghitung panjang maksimal continuous subarray yang strictly increasing dari array yang tersisa.

Ingat bahwa continuous subarray $A$ dengan indeks dari $l$ ke $r$ adalah $A[l…r] = A[l], A[l + 1], ..., A[r]$. Subarray $A[l...r]$ disebut strictly increasing jika $A[l] < A[l+1] < ... < A[r]$.

\subsection*{Format Masukan}

Baris pertama berisi sebuah bilangan bulat $N$ ($2 \leq N \leq 2 \times 10^{5}$) — banyaknya elemen pada subarray $A$. Baris kedua berisi $N$ bilangan bulat $A[1], A[2], ..., A[N] (1 \leq A[i] \leq 10^{9})$, di mana $A[i]$ adalah elemen ke-$i$ dari $A$.

\subsection*{Format Keluaran}

Cetak sebuah bilangan bulat yaitu panjang maksimal continuous subarray yang strictly increasing dari array $A$ setelah menghapus paling banyak satu elemen.

\begin{multicols}{2}
\subsection*{Contoh Masukan 1}
\begin{lstlisting}
5
1 2 5 3 4
\end{lstlisting}
\columnbreak
\subsection*{Contoh Keluaran 1}
\begin{lstlisting}
4
\end{lstlisting}
\vfill
\null
\end{multicols}


\subsection*{Penjelasan}

Pada contoh ini, Anda bisa menghapus $A[3] = 5$. Kemudian array berubah menjadi [1,2,3,4] dan panjang continuous subarray yang strictly increasing adalah 4.

\end{document}