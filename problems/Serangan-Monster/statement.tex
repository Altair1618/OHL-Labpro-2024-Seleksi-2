\documentclass{article}

\usepackage{geometry}
\usepackage{amsmath}
\usepackage{graphicx, eso-pic}
\usepackage{listings}
\usepackage{hyperref}
\usepackage{multicol}
\usepackage{fancyhdr}
\pagestyle{fancy}
\fancyhf{}
\hypersetup{ colorlinks=true, linkcolor=black, filecolor=magenta, urlcolor=cyan}
\geometry{ a4paper, total={170mm,257mm}, top=10mm, right=20mm, bottom=20mm, left=20mm}
\setlength{\parindent}{0pt}
\setlength{\parskip}{0.3em}
\renewcommand{\headrulewidth}{0pt}

\rfoot{\thepage}
\fancyhf{} % sets both header and footer to nothing
\renewcommand{\headrulewidth}{0pt}
\lfoot{\textbf{OHL Labpro 2024}}
\pagenumbering{gobble}

\fancyfoot[CE,CO]{\thepage}
\lstset{
    basicstyle=\ttfamily\small,
    columns=fixed,
    extendedchars=true,
    breaklines=true,
    tabsize=2,
    prebreak=\raisebox{0ex}[0ex][0ex]{\ensuremath{\hookleftarrow}},
    frame=none,
    showtabs=false,
    showspaces=false,
    showstringspaces=false,
    prebreak={},
    keywordstyle=\color[rgb]{0.627,0.126,0.941},
    commentstyle=\color[rgb]{0.133,0.545,0.133},
    stringstyle=\color[rgb]{01,0,0},
    captionpos=t,
    escapeinside={(\%}{\%)}
}

\begin{document}

\begin{center}

    
    \section*{Serangan Monster} % ganti judul soal

    \begin{tabular}{ | c c | }
        \hline
        Batas Waktu  & 1s \\    % jangan lupa ganti time limit
        Batas Memori & 256MB \\  % jangan lupa ganti memory limit
        \hline
    \end{tabular}
\end{center}

\subsection*{Deskripsi}

Markas OWCA sedang diserang. Dr. Asep Spakbor tiba-tiba meluncurkan $N$ monster yang masing-masing memiliki nyawa $H$ dan kekuatan $P$.

Beruntungnya, Dr. Agus Heisenberg telah membuat sebuah senjata mematikan yang dapat menyerang seluruh monster sekaligus yang mengurangi nyawa sebesar $K$. Namun, senjata ini memiliki kelemahan dimana dalam setiap pemakaian kekuatannya akan berkurang sebesar kekuatan terkecil dari seluruh monster yang masih hidup.

Tentukan apakah markas OWCA akan berhasil terlindungi dengan senjata itu

\subsection*{Format Masukan}

Baris pertama terdiri dari satu bilangan bulat positif $T$ ($1 \leq N \leq 100$)  yang banyak kasus.

Setiap kasus terdiri atas 3 baris. Baris pertama terdiri dari dua bilangan bulat positif $N$ ($1 \leq N \leq 10^{5}$) dan $K$ ($1 \leq K \leq 10^{5}$) yang masing-masing menyatakan jumlah monster dan kekuatan awal dari senjata Dr. Agus.
$2$ baris berikutnya berisi $N$ nilai $H_i$ ($1 \leq X_i \leq 10^{9}$) dan $P_i$ ($1 \leq Y_i \leq 10^{9}$) yang masing-masing menyatakan nyawa monster dan kekuatan monster.

Jumlah N pada seluruh kasus tidak akan melebihi $10^5$

\subsection*{Format Keluaran}

Keluarkan "YES" (tanpa tanda petik) apabila seluruh monster mati dan "NO" untuk sebaliknya.

\begin{multicols}{2}
\subsection*{Contoh Masukan 1}
\begin{lstlisting}
2
3 7
17 5 13
2 7 1
3 4
5 5 5
4 4 4
\end{lstlisting}
\columnbreak
\subsection*{Contoh Keluaran 1}
\begin{lstlisting}
YES
NO
\end{lstlisting}
\vfill
\null
\end{multicols}


\subsection*{Penjelasan}

Pada contoh 1, pada serangan pertama nyawa dan kekuatan senjata akan berubah menjadi berikut

H: [10, 0, 6]
K: 7 - 1 = 6

Lalu, setelah serangan kedua

H: [4, 0, 0]
K: 5 - 2 = 3

Lalu, setelah serangan ketiga

H: [1, 0, 0]
K: 3 - 2 = 1

Lalu, setelah serangan keempat

H: [0, 0, 0]

Karena seluruh monster mati maka keluarannya adalah YES.

\end{document}