\documentclass{article}

\usepackage{geometry}
\usepackage{amsmath}
\usepackage{graphicx, eso-pic}
\usepackage{listings}
\usepackage{hyperref}
\usepackage{multicol}
\usepackage{fancyhdr}
\pagestyle{fancy}
\fancyhf{}
\hypersetup{ colorlinks=true, linkcolor=black, filecolor=magenta, urlcolor=cyan}
\geometry{ a4paper, total={170mm,257mm}, top=10mm, right=20mm, bottom=20mm, left=20mm}
\setlength{\parindent}{0pt}
\setlength{\parskip}{0.3em}
\renewcommand{\headrulewidth}{0pt}

\rfoot{\thepage}
\fancyhf{} % sets both header and footer to nothing
\renewcommand{\headrulewidth}{0pt}
\lfoot{\textbf{OHL Labpro 2024}}
\pagenumbering{gobble}

\fancyfoot[CE,CO]{\thepage}
\lstset{
    basicstyle=\ttfamily\small,
    columns=fixed,
    extendedchars=true,
    breaklines=true,
    tabsize=2,
    prebreak=\raisebox{0ex}[0ex][0ex]{\ensuremath{\hookleftarrow}},
    frame=none,
    showtabs=false,
    showspaces=false,
    showstringspaces=false,
    prebreak={},
    keywordstyle=\color[rgb]{0.627,0.126,0.941},
    commentstyle=\color[rgb]{0.133,0.545,0.133},
    stringstyle=\color[rgb]{01,0,0},
    captionpos=t,
    escapeinside={(\%}{\%)}
}

\begin{document}

\begin{center}

    
    \section*{Labirin Berportal} % ganti judul soal

    \begin{tabular}{ | c c | }
        \hline
        Batas Waktu  & 2s \\    % jangan lupa ganti time limit
        Batas Memori & 256MB \\  % jangan lupa ganti memory limit
        \hline
    \end{tabular}
\end{center}

\subsection*{Deskripsi}

Purry dan agen-agen lainnya sedang terperangkap dalam labirin buatan Dr. Asep Spakbor. Meskipun labirinnya sangat rumit, Purry berhasil mendapatkan peta labirin setelah melakukan \textit{exploit} pada komputer Dr. Asep Spakbor. 

Setelah melakukan analisis pada peta, Purry telah menentukan jalur yang akan mereka lalui. Namun, pada jalur yang mereka lalui, ada $N$ portal yang dapat mengembalikan mereka ke suatu lokasi yang telah mereka lalui. Portal ke-$i$ berada dalam posisi $x_i$ dan akan mengembalikan Purry ke posisi $y_i$ dimana $y_i < x_i$.

Setiap portal dapat berada dalam 2 \textit{state} yaitu aktif atau tidak aktif. Apabila Purry melalui portal yang aktif, Purry akan kembali ke posisi $y_i$ dan portalnya menjadi tidak aktif. Sebaliknya, apabila Purry melalui portal yang tidak aktif, Purry dapat bergerak ke posisi $x_i + 1$ dan portal menjadi aktif.

Tentukan berapa jarak yang harus ditempuh Purry dan agen OWCA lainnya untuk mencapai pintu keluar yang berada di posisi $K$ dari posisi awal mereka yaitu $0$. Karena jawabannya dapat menjadi sangat besar, keluarkan dalam modulo $998244353$.

\subsection*{Format Masukan}

Baris pertama terdiri dari dua bilangan bulat positif $N$ ($1 \leq N \leq 2 \times 10^{5}$) dan $K$ ($2 \times N < K \leq 10^{9}$) yang masing-masing menyatakan banyak portal dan lokasi pintu keluar.

$N$ baris berikutnya berisi 3 bilangan $x_i$ ($2 \leq x_i < K$), $y_i$ ($1 \leq y_i < x_i$), dan $b_i$ ($0 \leq b_i \leq 1$) yang masing-masing menyatakan lokasi portal, posisi akhir Purry setelah teleportasi, dan \textit{state} awal portal tersebut (0 menyatakan tidak aktif dan 1 menyatakan aktif)

Nilai $x_i$ akan selalu terurut menaik yaitu $x_1 < x_2 < x_3 < ... < x_n$. Nilai $x_i$ dan $y_i$ dipastikan seluruhnya unik (tidak ada yang memiliki nilai sama).

\subsection*{Format Keluaran}

Keluarkan jarak yang harus ditempuh Purry untuk mencapai pintu keluar dalam jalur tersebut dari posisi awal $0$. Karena jawabannya dapat menjadi sangat besar, keluarkan dalam modulo $998244353$.

\begin{multicols}{2}
\subsection*{Contoh Masukan 1}
\begin{lstlisting}
4 9
3 2 1
6 5 0
7 4 0
8 1 1
\end{lstlisting}
\columnbreak
\subsection*{Contoh Keluaran 1}
\begin{lstlisting}
23
\end{lstlisting}
\vfill
\null
\end{multicols}

\begin{multicols}{2}
\subsection*{Contoh Masukan 2}
\begin{lstlisting}
5 1000000000
199999999 100000000 1
599999999 400000000 0
799999999 300000000 0
899999999 700000000 1
999999999 500000000 0
\end{lstlisting}
\columnbreak
\subsection*{Contoh Keluaran 2}
\begin{lstlisting}
3511290
\end{lstlisting}
\vfill
\null
\end{multicols}

\subsection*{Penjelasan}

Berikut adalah penjelasan untuk contoh 1: \\
- Posisi: 0, Total Jarak: 0. \\
- Posisi: 3, Total Jarak: 3. Disini Purry akan terteleportasi ke posisi 2 karena portal ini aktif. \\
- Posisi: 2, Total Jarak: 3. \\
- Posisi: 3, Total Jarak: 4. Disini Purry tidak akan terteleportasi karena portal tidak lagi aktif. \\
- Posisi: 8, Total Jarak: 9. \\
- Posisi: 1, Total Jarak: 9. \\
- Posisi: 3, Total Jarak: 11. \\
- Posisi: 2, Total Jarak: 11. \\
- Posisi: 6, Total Jarak: 15. \\
- Posisi: 5, Total Jarak: 15. \\
- Posisi: 7, Total Jarak: 17. \\
- Posisi: 4, Total Jarak: 17. \\
- Posisi: 6, Total Jarak: 19. \\
- Posisi: 5, Total Jarak: 19. \\
- Posisi: 9, Total Jarak: 23. \\

Maka, total jarak yang harus ditempuh Purry adalah 23.

\end{document}