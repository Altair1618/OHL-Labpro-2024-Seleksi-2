\documentclass{article}

\usepackage{geometry}
\usepackage{amsmath}
\usepackage{graphicx, eso-pic}
\usepackage{listings}
\usepackage{hyperref}
\usepackage{multicol}
\usepackage{fancyhdr}
\pagestyle{fancy}
\fancyhf{}
\hypersetup{ colorlinks=true, linkcolor=black, filecolor=magenta, urlcolor=cyan}
\geometry{ a4paper, total={170mm,257mm}, top=10mm, right=20mm, bottom=20mm, left=20mm}
\setlength{\parindent}{0pt}
\setlength{\parskip}{0.3em}
\renewcommand{\headrulewidth}{0pt}

\rfoot{\thepage}
\fancyhf{} % sets both header and footer to nothing
\renewcommand{\headrulewidth}{0pt}
\lfoot{\textbf{OHL Labpro 2024}}
\pagenumbering{gobble}

\fancyfoot[CE,CO]{\thepage}
\lstset{
    basicstyle=\ttfamily\small,
    columns=fixed,
    extendedchars=true,
    breaklines=true,
    tabsize=2,
    prebreak=\raisebox{0ex}[0ex][0ex]{\ensuremath{\hookleftarrow}},
    frame=none,
    showtabs=false,
    showspaces=false,
    showstringspaces=false,
    prebreak={},
    keywordstyle=\color[rgb]{0.627,0.126,0.941},
    commentstyle=\color[rgb]{0.133,0.545,0.133},
    stringstyle=\color[rgb]{01,0,0},
    captionpos=t,
    escapeinside={(\%}{\%)}
}

\begin{document}

\begin{center}
    \section*{Kontes Kecantikan di Danville}
    
    \begin{tabular}{ | c c | }
        \hline
        Batas Waktu  & 1s \\ 
        Batas Memori & 256MB \\ 
        \hline
    \end{tabular}
\end{center}

\subsection*{Deskripsi}

Wali kota Danville memutuskan untuk mengadakan kontes kecantikan di kota tersebut. Setiap peserta kontes diberi penilaian berdasarkan kecantikan mereka. Tugas Anda adalah membantu mereka menemukan peserta dengan penilaian ke-K tertinggi di antara semua peserta.

\subsection*{Format Masukan}

Masukan terdiri dari beberapa baris:
\begin{itemize}
    \item Baris pertama berisi dua bilangan bulat $N$ $(1 \leq N \leq 100000)$ dan $K$ $(1 \leq K \leq N)$, masing-masing menyatakan jumlah peserta dan penilaian ke berapa yang ingin dicari.
    \item Baris berikutnya berisi $N$ bilangan bulat yang merupakan penilaian dari setiap peserta kontes.
\end{itemize}

\subsection*{\textbf{Kebutuhan Tambahan}}
\begin{itemize}
    \item Harus menggunakan varian struktur data PrioQueue
    \item Tidak boleh menggunakan fungsi sorting
\end{itemize}

\subsection*{Format Keluaran}

Keluaran adalah sebuah bilangan bulat yang merupakan penilaian ke-K tertinggi dari peserta kontes.

\begin{multicols}{2}
\subsection*{Contoh Masukan}
\begin{lstlisting}
6 2
3 2 1 5 6 4
\end{lstlisting}
\columnbreak
\subsection*{Contoh Keluaran}
\begin{lstlisting}
5
\end{lstlisting}
\end{multicols}

\subsection*{Penjelasan}

Dalam contoh ini, ada 6 peserta dengan penilaian sebagai berikut: 3, 2, 1, 5, 6, 4. Penilaian tertinggi kedua adalah 5.

\begin{multicols}{2}
\subsection*{Contoh Masukan 2}
\begin{lstlisting}
9 4
3 2 3 1 2 4 5 5 6
\end{lstlisting}
\columnbreak
\subsection*{Contoh Keluaran 2}
\begin{lstlisting}
4
\end{lstlisting}
\end{multicols}

\subsection*{Penjelasan}

Dalam contoh kedua, ada 9 peserta dengan penilaian sebagai berikut: 3, 2, 3, 1, 2, 4, 5, 5, 6. Penilaian tertinggi keempat adalah 4.

\end{document}
