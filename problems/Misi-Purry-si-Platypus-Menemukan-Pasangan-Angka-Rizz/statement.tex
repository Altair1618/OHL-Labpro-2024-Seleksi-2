\documentclass{article}

\usepackage{geometry}
\usepackage{amsmath}
\usepackage{graphicx, eso-pic}
\usepackage{listings}
\usepackage{hyperref}
\usepackage{multicol}
\usepackage{fancyhdr}
\pagestyle{fancy}
\fancyhf{}
\hypersetup{ colorlinks=true, linkcolor=black, filecolor=magenta, urlcolor=cyan}
\geometry{ a4paper, total={170mm,257mm}, top=10mm, right=20mm, bottom=20mm, left=20mm}
\setlength{\parindent}{0pt}
\setlength{\parskip}{0.3em}
\renewcommand{\headrulewidth}{0pt}

\rfoot{\thepage}
\fancyhf{} % sets both header and footer to nothing
\renewcommand{\headrulewidth}{0pt}
\lfoot{\textbf{OHL Labpro 2024}}
\pagenumbering{gobble}

\fancyfoot[CE,CO]{\thepage}
\lstset{
    basicstyle=\ttfamily\small,
    columns=fixed,
    extendedchars=true,
    breaklines=true,
    tabsize=2,
    prebreak=\raisebox{0ex}[0ex][0ex]{\ensuremath{\hookleftarrow}},
    frame=none,
    showtabs=false,
    showspaces=false,
    showstringspaces=false,
    prebreak={},
    keywordstyle=\color[rgb]{0.627,0.126,0.941},
    commentstyle=\color[rgb]{0.133,0.545,0.133},
    stringstyle=\color[rgb]{01,0,0},
    captionpos=t,
    escapeinside={(\%}{\%)}
}

\begin{document}

\begin{center}
    \section*{Misi Purry si Platypus: Menemukan Pasangan Angka Rizz!} % ganti judul soal

    \begin{tabular}{ | c c | }
        \hline
        Batas Waktu  & 4s \\    % jangan lupa ganti time limit
        Batas Memori & 256MB \\  % jangan lupa ganti memory limit
        \hline
    \end{tabular}
\end{center}

\subsection*{Deskripsi}

Pada suatu hari, Purry si Platypus lagi jalan-jalan dan tiba-tiba ketemu puzzle yang bikin otak mikir keras kayak sigma! Ada empat angka positif $a, b, c, d$ dengan $a < c$ dan $b < d$. Tugas Purry adalah mencari sepasang angka $x$ dan $y$ yang memenuhi syarat berikut:

\begin{itemize}
    \item $a < x \leq c$
    \item $b < y \leq d$
    \item $x \cdot y$ habis dibagi oleh $a \cdot b$
\end{itemize}

Purry si Platypus harus menemukan pasangan angka $x$ dan $y$ yang sesuai, tapi mungkin aja pasangan yang dimaksud nggak ada. Sus banget, kan?

Bantuin Purry si Platypus buat nemuin angka-angka itu biar dia bisa lanjut jalan-jalan lagi! Sheesh, puzzle ini bener-bener rizz banget! Kalau berhasil, Purry bakal flex ke semua temennya!

\subsection*{Format Masukan}

Baris pertama input berisi sebuah bilangan bulat $t$ ($1 \leq t \leq 10$), jumlah kasus uji.

Setiap kasus uji berisi empat bilangan bulat $a, b, c,$ dan $d$ ($1 \leq a < c \leq 10^9, 1 \leq b < d \leq 10^9$).


\subsection*{Format Keluaran}

Untuk setiap kasus uji, cetak sepasang angka $x$ dan $y$ di mana $a < x \leq c$ dan $b < y \leq d$ sedemikian sehingga $x \cdot y$ habis dibagi oleh $a \cdot b$. Jika terdapat beberapa jawaban yang memenuhi syarat, cetak salah satunya. Jika tidak ada pasangan angka yang memenuhi syarat tersebut, maka cetak $-1$ $-1$.

\\

\begin{multicols}{2}
\subsection*{Contoh Masukan}
\begin{lstlisting}
10
1 1 2 2
8 9 14 18
3 4 5 8
36 60 48 87
12 20 14 25
1024 729 373248 730
1024 729 373247 730
5040 20310 40319 1000000000
999999999 999999999 1000000000 1000000000
168435456 168435456 1000000000 1000000000
\end{lstlisting}
\columnbreak
\subsection*{Contoh Keluaran}
\begin{lstlisting}
2 2
12 18
4 6
-1 -1
-1 -1
373248 730
-1 -1
5041 102362400
-1 -1
170459136 998614806
\end{lstlisting}
\end{multicols}

\end{document}