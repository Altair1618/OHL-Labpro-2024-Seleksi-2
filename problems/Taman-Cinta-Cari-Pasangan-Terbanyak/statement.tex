\documentclass{article}

\usepackage{geometry}
\usepackage{amsmath}
\usepackage{graphicx, eso-pic}
\usepackage{listings}
\usepackage{hyperref}
\usepackage{multicol}
\usepackage{fancyhdr}
\pagestyle{fancy}
\fancyhf{}
\hypersetup{ colorlinks=true, linkcolor=black, filecolor=magenta, urlcolor=cyan}
\geometry{ a4paper, total={170mm,257mm}, top=10mm, right=20mm, bottom=20mm, left=20mm}
\setlength{\parindent}{0pt}
\setlength{\parskip}{0.3em}
\renewcommand{\headrulewidth}{0pt}

\rfoot{\thepage}
\fancyhf{} % sets both header and footer to nothing
\renewcommand{\headrulewidth}{0pt}
\lfoot{\textbf{OHL Labpro 2024}}
\pagenumbering{gobble}

\fancyfoot[CE,CO]{\thepage}
\lstset{
    basicstyle=\ttfamily\small,
    columns=fixed,
    extendedchars=true,
    breaklines=true,
    tabsize=2,
    prebreak=\raisebox{0ex}[0ex][0ex]{\ensuremath{\hookleftarrow}},
    frame=none,
    showtabs=false,
    showspaces=false,
    showstringspaces=false,
    prebreak={},
    keywordstyle=\color[rgb]{0.627,0.126,0.941},
    commentstyle=\color[rgb]{0.133,0.545,0.133},
    stringstyle=\color[rgb]{01,0,0},
    captionpos=t,
    escapeinside={(\%}{\%)}
}

\begin{document}

\begin{center}
    \section*{Taman Cinta Danvile - Cari Penilaian Pasangan Terbanyak}
    
    \begin{tabular}{ | c c | }
        \hline
        Batas Waktu  & 1s \\ 
        Batas Memori & 256MB \\ 
        \hline
    \end{tabular}
\end{center}

\subsection*{Deskripsi}

Di Taman Cinta Danville, setiap peserta kontes ingin mencari pasangan sebanyak mungkin dengan jumlah penilaian tertentu. Setiap peserta, baik perempuan maupun laki-laki, memiliki nomor pada punggungnya yang bernilai antara $1$ hingga $K$. Panitia acara menginginkan untuk mencari pasangan (baik laki-laki dan perempuan, atau sesama jenis) yang nilainya berjumlah $K$. Jika mereka bernilai $K$, mereka akan dikeluarkan dari taman. 

Tugas Anda adalah untuk membantu panitia mencari pasangan sebanyak mungkin dengan jumlah penilaian tepat $K$.

\subsection*{Format Masukan}

Masukan terdiri dari beberapa baris:
\begin{itemize}
    \item Baris pertama berisi dua bilangan bulat $N$ $(1 \leq N \leq 10^6)$ dan $K$ $(1 \leq K \leq 10^9)$, masing-masing menyatakan jumlah peserta dan penilaian pasangan yang ingin dicari.
    \item Baris berikutnya berisi $N$ bilangan bulat yang merupakan penilaian dari setiap peserta kontes.
\end{itemize}

\subsection*{Format Keluaran}

Keluaran adalah sebuah bilangan bulat yang merupakan jumlah operasi maksimum yang dapat Anda lakukan pada array.

\begin{multicols}{2}
\subsection*{Contoh Masukan}
\begin{lstlisting}
5 5
1 2 3 4 3
\end{lstlisting}
\columnbreak
\subsection*{Contoh Keluaran}
\begin{lstlisting}
2
\end{lstlisting}
\end{multicols}

\subsection*{Penjelasan}

Dalam contoh ini, ada 5 peserta dengan penilaian sebagai berikut: 1, 2, 3, 4, 3. Pasangan-pasangan yang mungkin adalah:\\
- 1 + 4 \\
- 2 + 3 \\
Jumlah maksimum operasi yang dapat dilakukan adalah 2.

\begin{multicols}{2}
\subsection*{Contoh Masukan 2}
\begin{lstlisting}
4 2
1 1 1 1
\end{lstlisting}
\columnbreak
\subsection*{Contoh Keluaran 2}
\begin{lstlisting}
2
\end{lstlisting}
\end{multicols}

\subsection*{Penjelasan}

Dalam contoh kedua, ada 4 peserta dengan penilaian sebagai berikut: 1, 1, 1, 1. Pasangan-pasangan yang mungkin adalah:\\
- 1 + 1 \\
- 1 + 1 \\
Jumlah maksimum operasi yang dapat dilakukan adalah 2.

\end{document}
