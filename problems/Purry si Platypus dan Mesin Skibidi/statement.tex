\documentclass{article}

\usepackage{geometry}
\usepackage{amsmath}
\usepackage{graphicx, eso-pic}
\usepackage{listings}
\usepackage{hyperref}
\usepackage{multicol}
\usepackage{fancyhdr}
\pagestyle{fancy}
\fancyhf{}
\hypersetup{ colorlinks=true, linkcolor=black, filecolor=magenta, urlcolor=cyan}
\geometry{ a4paper, total={170mm,257mm}, top=10mm, right=20mm, bottom=20mm, left=20mm}
\setlength{\parindent}{0pt}
\setlength{\parskip}{0.3em}
\renewcommand{\headrulewidth}{0pt}

\rfoot{\thepage}
\fancyhf{} % sets both header and footer to nothing
\renewcommand{\headrulewidth}{0pt}
\lfoot{\textbf{OHL Labpro 2024}}
\pagenumbering{gobble}

\fancyfoot[CE,CO]{\thepage}
\lstset{
    basicstyle=\ttfamily\small,
    columns=fixed,
    extendedchars=true,
    breaklines=true,
    tabsize=2,
    prebreak=\raisebox{0ex}[0ex][0ex]{\ensuremath{\hookleftarrow}},
    frame=none,
    showtabs=false,
    showspaces=false,
    showstringspaces=false,
    prebreak={},
    keywordstyle=\color[rgb]{0.627,0.126,0.941},
    commentstyle=\color[rgb]{0.133,0.545,0.133},
    stringstyle=\color[rgb]{01,0,0},
    captionpos=t,
    escapeinside={(\%}{\%)}
}

\begin{document}

\begin{center}
    \section*{Purry si Platypus dan Mesin Skibidi} % ganti judul soal

    \begin{tabular}{ | c c | }
        \hline
        Batas Waktu  & 1s \\    % jangan lupa ganti time limit
        Batas Memori & 256MB \\  % jangan lupa ganti memory limit
        \hline
    \end{tabular}
\end{center}

\subsection*{Deskripsi}

Pada suatu hari, Purry si Platypus ketemu mesin yang bikin ketawa-ketawa kayak bintang skibidi! Mesin ini punya layar yang suuuper gede dan cuma ada satu tombol. Waktu pertama kali Purry nemu mesinnya, layarnya cuma nunjukin huruf A doang. Trus, pas dia pencet tombolnya, hurufnya berubah jadi B. No cap, beberapa kali dia pencet tombol lagi, katanya berubah dari B jadi BA, terus jadi BAB, terus jadi BABBA... Sheesh!

Purry si Platypus jadi ngerti kalau mesin ini ngubah katanya kayak gini: setiap huruf B jadi BA dan setiap huruf A jadi B. Flex banget kan mesinnya? Tapi terus Purry jadi kepo dan punya pertanyaan sigma banget! Setelah $N$ kali pencet tombol, bakal ada berapa banyak huruf A dan huruf B di layar ya?

\subsection*{Format Masukan}

Baris pertama terdiri dari satu bilangan bulat $N$ ($1 \leq N \leq 45$), yaitu jumlah Purry si Platypus menekan tombol.

\subsection*{Format Keluaran}

Keluarkan satu baris terdiri dari dua bilangan bulat yang dipisahkan oleh spasi, yaitu jumlah huruf $A$ dan jumlah huruf $B$.
\\

\begin{multicols}{2}
\subsection*{Contoh Masukan 1}
\begin{lstlisting}
1
\end{lstlisting}
\subsection*{Contoh Masukan 2}
\begin{lstlisting}
4
\end{lstlisting}
\subsection*{Contoh Masukan 3}
\begin{lstlisting}
10
\end{lstlisting}
\columnbreak
\subsection*{Contoh Keluaran 1}
\begin{lstlisting}
0 1
\end{lstlisting}
\subsection*{Contoh Keluaran 2}
\begin{lstlisting}
2 3
\end{lstlisting}
\subsection*{Contoh Keluaran 3}
\begin{lstlisting}
34 55
\end{lstlisting}
\end{multicols}

\end{document}